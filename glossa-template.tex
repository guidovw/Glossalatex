% !TEX TS-program = xelatexmk
% glossa-template.tex
% Copyright 2016 Guido Vanden Wyngaerd
%
% This work may be distributed and/or modified under the
% conditions of the LaTeX Project Public License.
% The latest version of this license is in
%   http://www.latex-project.org/lppl.txt
% and version 1.3 or later is part of all distributions of LaTeX
% version 2005/12/01 or later.
%
% This work has the LPPL maintenance status `maintained'.
% 
% The Current Maintainer of this work is 
% Guido Vanden Wyngaerd (guido.vandenwyngaerd@kuleuven.be).
%
% This work consists of the files 
% glossa.cls
% glossa.bst
% gl-authoryear-comp.cbx
% biblatex-gl.bbx
% glossa-template.tex
% glossa.png
%
% The files of the work are derived from the Semantics & Pragmatics style files
% by Kai von Fintel, Christopher Potts, and Chung-chieh Shan
% All changes are documented on the github repository 
% https://github.com/guidovw/Glossalatex.

\documentclass[charis,linguex]{glossa}
% possible options:
% [times] for Times font (default if no option is chosen)
% [cm] for Computer Modern font
% [lucida] for Lucida font (not freely available)
% [brill] open type font, freely downloadable for non-commercial use from http://www.brill.com/about/brill-fonts; requires xetex
% [charis] for CharisSIL font, freely downloadable from http://software.sil.org/charis/
% for the Brill an CharisSIL fonts, you have to use the XeLatex typesetting engine (not pdfLatex)
% for headings, tables, captions, etc., Fira Sans is used: https://www.fontsquirrel.com/fonts/fira-sans
% [biblatex] for using biblatex (the default is natbib, do not load the natbib package in this file, it is loaded automatically via the document class glossa.cls)
% [linguex] loads the linguex example package
% !! a note on the use of linguex: in glossed examples, the third line of the example (the translation) needs to be prefixed with \glt. This is to allow a first line with the name of the language and the source of the example. See example (2) in the text for an illustration.
% !! a note on the use of bibtex: for PhD dissertations to typeset correctly in the references list, the Address field needs to contain the city (for US cities in the format "Santa Cruz, CA")

%\addbibresource{sample.bib}
% the above line is for use with biblatex
% replace this by the name of your bib-file (extension .bib is required)
% comment out if you use natbib/bibtex

\let\B\relax %to resolve a conflict in the definition of these commands between xyling and xunicode (the latter called by fontspec, called by charis)
\let\T\relax
\usepackage{xyling} %for trees; the use of xyling with the CharisSIL font produces poor results in the branches. This problem does not arise with the packages qtree or forest.
\usepackage[linguistics]{forest} %for nice trees!

% \pdf* commands provide metadata for the PDF output. ASCII characters only!
\pdfauthor{Waltraud Paul \& Guido Vanden Wyngaerd}
\pdftitle{Glossa submission guidelines}
\pdfkeywords{Full keyword list, separated, by, commas}

\title[Glossa guidelines]{Glossa submission guidelines\\ \bigskip \large Word count: 4720}
% Optional short title inside square brackets, for the running headers.

\author[Paul \& Vanden Wyngaerd]% short form of the author names for the running header. If no short author is given, no authors print in the headers.
{%as many authors as you like, each separated by \AND.
  \spauthor{Waltraud Paul\\
  \institute{CRLAO, CNRS-EHESS-INALCO}\\
  \small{%105, Bd. Raspail, 75005 Paris\\
  waltraud.paul@ehess.fr}
  }
  \AND
  \spauthor{Guido Vanden Wyngaerd \\
  \institute{KU Leuven}\\
  \small{%Warmoesberg 26, 1000 Brussel\\
  guido.vandenwyngaerd@kuleuven.be}
  }%
}

\begin{document}


\maketitle


\begin{abstract}
This document provides a full overview of the information relating to Glossa submissions. This information includes (i) the author guidelines, and (ii) the stylesheet. So as to provide instruction both by example and by rule, this document has been formatted in accordance with the stylesheet it contains.
\end{abstract}

\begin{keywords}
  stylesheet; Glossa; latex template
\end{keywords}


\section{Author guidelines}
\subsection{Submission information}

%\sloppy
Submissions should be made electronically through the \href{https://www.glossa-journal.org}{Glossa website}.

Prior to submission, please add a word count (including footnotes and references) directly under the paper title. Then convert your paper into \textbf{a single PDF file, containing all tables and figures}. Non-PDF files or separately provided files may be returned prior to review. Separate image files may be requested if the submission is accepted for publication.

Please ensure that you consider the following guidelines when preparing your manuscript. Failure to do so may delay the processing of your submission. A downloadable version of the style guide is available \href{https://github.com/guidovw/Glossalatex/blob/master/glossa-template.pdf}{\texttt{here}}. Text formatting in accordance with  the stylesheet is required for the accepted version only.


For \LaTeX\ submissions, please download the Latex resources \href{https://github.com/guidovw/Glossalatex}{\texttt{here}}.

\textbf{NOTE}: \emph{All files must be anonymised during the initial submission (including information in the file properties). Only after editorial acceptance should you add author details to the manuscript files.}

Once a submission has been completed, the submitting author is able to fully track the status of the paper and complete requested revisions via their online profile.
%\fussy

\subsection{Article types}

\subsubsection{Research articles}

Research articles must describe the outcomes and application of unpublished original research. These should make a substantial contribution to knowledge and understanding in the subject matter and should be supported by relevant figures and tabulated data. Research articles must initially be no more than \textbf{13,000 words} in length, and can be extended to a maximum of 15,000 words after revisions. Authors are allowed to add appendices with supplementary material that will be hosted separately from the article itself, and receive their own, properly referenced, DOI. These materials will not be typeset (see the below 'Structure' section on how to provide supplementary/data files).

\subsubsection{Overview articles}

Overview articles must describe the state-of-the art in a given subdiscipline or a specific topic in linguistics. They should be very accessible, aimed at an audience of MA students or interested colleagues. Overview articles must be no more than \textbf{13,000 words} in length. Authors are allowed to add appendices with supplementary material that will be hosted separately from the article itself, and receive their own, properly referenced, DOI. These materials will not be typeset.  See below on how to provide supplementary/data files.

\subsubsection{Book reviews}

Book reviews present critical appraisals of recent books in linguistics, with a preference for monographs, handbooks, and grammars. They can cover topics such as current controversies or the historical development of studies as well as issues of regional or temporal focus. Papers should critically engage with the relevant body of extant literature. Book reviews should be no longer than \textbf{3,000 words} in length.

\subsubsection{Review articles}

Review articles present longer critical appraisals of one or more recent books containing an original contribution or perspective on the book(s) reviewed. Review articles will be reviewed by the editors and/ or members of the editorial board. Review articles should be no longer than \textbf{6,000 words} in length.

\subsubsection{Squibs}

Squibs are short notes (\textbf{5,000 words} max.) that make a scintillating point by calling attention to a theoretically unexpected observation about language, without the need for a developed analysis or solution.

\subsubsection{Special Collections}

Special Collections (SCs) are collections of papers devoted to a particular topic, and edited by a team of guest editors. This usually means that authors are invited by the guest editors to submit to an SC. Nevertheless, contributions to special collections are subject to the usual editorial processes of blind peer review.

The main concern for SCs is to achieve a strong thematic unity, avoiding the impression of conference proceedings. The Glossa policy on SCs is to prefer small SCs with papers that constitute a tight fit, over very general ones with contributions that are only loosely connected thematically. The papers in an SC should be strongly complementary, and ‘talk’ to each other. In short, a SC should avoid the impression that it is a mere collection of standalone papers on the same topic from different insular perspectives.

More information about special collections can be found on the \href{https://www.glossa-journal.org/site/editorial-policies/}{Editorial Policies} page.



\subsubsection{Word limits}

All word limits mentioned above include referencing and citation, but they exclude appendices, data files and other supplementary material. Please note that if you have data or supplementary files, they should be treated as outlined in the section \textbf{data availability/supplementary files} below, and not as part of the main submission file.


\subsection{Permissions}

The author is responsible for obtaining all permissions required prior to submission of the manuscript. Permission and owner details should be mentioned for all third-party content included in the submission or used in the research.

If a method or tool is introduced in the study, including software, questionnaires, and scales, the license this is available under and any requirement for permission for use should be stated. If an existing method or tool is used in the research, it is the author's responsibility to check the license and obtain the necessary permissions. %Statements confirming that permission was granted should be included in the Materials and Methods section.


\section{Style sheet}\label{ss}

The Glossa style sheet is based on the \href{http://www.eva.mpg.de/linguistics/past-research-resources/resources/generic-style-rules.html}{The Generic Style Rules for Linguistics} (December 2014 version), developed under a CC-BY licence by Martin Haspelmath. It was slightly modified for Glossa by Waltraud Paul and Guido Vanden Wyngaerd in November 2015, and again in May 2021.

\subsection{Structure}

\subsubsection{Title page}

The title should not contain any capitalisation apart from the first word and words that need capitals in any context. In the final version of the accepted paper, the title is followed by the first and last name of the author(s), their affiliation, and e-mail. First names should not include only initials.

Anonymisation: The names of all authors, affiliations, contact details, biography (optional) and the corresponding author details must be completed online as part of the submission process but should not be added to the submitted files until after editorial acceptance.

\subsubsection{Abstract}

Articles must have the main text prefaced by an abstract of no more than 250 words summarising the main arguments and conclusions of the article. A list of up to six key words may be placed below the abstract (optional). The abstract and keywords should also be added to the metadata when making the initial online submission. The abstract is automatically attached to the email message inviting reviewers to review the paper.

\subsubsection{Main text}

Articles are subdivided into numbered sections (and possibly subsections, numbered 1.1 etc., and subsubsections, numbered 1.1.1 etc.), with a bold-faced heading in each case. The numbering always begins with 1, not 0. Section headings do not end with a period, and have no special capitalisation.

\subsubsection{Unnumbered sections}
The conclusion is the last numbered section. It may be followed by several (optional) unnumbered sections, in this order:
\begin{itemize}
\item Abbreviations
\item Data availability/Supplementary files
\item Ethics and consent
\item Funding information
\item Acknowledgements
\item Competing interests
\item Authors' contributions
\end{itemize}

\noindent Of these, only the Competing interests statement is mandatory, and, if your paper contains glossed examples, the Abbreviations section. More explanation on the content of these sections is provided below.

\subsubsection{References}
All references cited within the submission must be listed at the end of the main text file.

\subsection{Numbered examples and formulae}

Examples from languages other than English must \emph{all} be glossed (with word-by-word alignment) and translated, even if the translation seems obvious. The Leipzig Glossing Rules are recommended as basic guidelines, and can be found \href{http://www.eva.mpg.de/lingua/resources/glossing-rules.php}{here}. A full list of all the glosses used must be provided in the Abbreviations section. Example numbers are enclosed in parentheses, and left-aligned. Examples are numbered consecutively. When an earlier example is repeated, it gets a new number. Example sentences usually have normal capitalization at the beginning and normal punctuation. The gloss line has no capitalization and no punctuation.

\ex. \ag. Ich   kenne das Kind, dem du geholfen hast.\\
I.\textsc{nom} know the child.\textsc{acc} \textsc{dem.dat} you.\textsc{nom} helped have\\
\glt `I know the child that you helped.'
\bg. Ich kenne das Kind, dem du nicht geholfen hast. \\
I.\textsc{nom} know  the child.\textsc{acc} \textsc{dem.dat} you.\textsc{nom} \textsc{neg} helped   have\\
\glt `I know the child that you didn’t help.’

When the example is not a complete sentence, there is no capitalization and no full stop at the end. If the name of the language is added, the source of the example, or any extra information, this information must be added on an extra first line of the example (with the name of the language in italics).\footnote{Examples in footnotes are numbered with lower case Roman numerals enclosed between brackets:

\ex.
\a. Colorless green ideas sleep furiously.
\b. *The child seems sleeping.

More text can follow the example.}

\ex. \textit{German} \citep{coetsem:2000}\\ %optional line for the name of the language (italics), source, etc. Note the absence of \exg., instead use \ex. (and \a. \b. for subdivisions) when this optional line is present. Use \exg. etc. if this line is absent (as in the previous example)
\gll das Kind, dem du geholfen hast\\  %original foreign language example preceded by \gll
the child.\textsc{nom} \textsc{dem.dat} you.\textsc{nom}  helped have\\ %gloss line
\glt `the child that you helped' %translation, preceded by \glt

Ungrammatical examples can be given a parenthesized idiomatic translation. A literal translation may be given in parentheses after the idiomatic translation.

The use of any nonstandard layout in examples beyond what is illustrated above is strongly discouraged, as this will increase production time (and cost) of your paper, as well as increase the chances of the HTML version including errors in some browsers/screen sizes. If you feel an example needs additional explanation, try as much as possible to provide this in the text that goes with the example. If nonstandard layout is essential, then please raise this with the editorial team to discuss the options available.

Formulae must be proofed carefully by the author. Editors will not edit formulae. If special software has been used to create formulae, the way they are laid out is the way they will appear in the publication.

\subsection{Use of footnotes/endnotes}\label{fn}

Use footnotes rather than endnotes (we refer to these as ‘Notes’ in the online publication). These will appear at the bottom of each page. Notes should be used only where crucial clarifying information needs to be conveyed.

Avoid using notes for purposes of referencing; use in-text citations instead. If in-text citations cannot be used, a source can be cited as part of a note. Please insert the footnote marker after the end punctuation.

The footnote reference number normally follows a period or a comma, though exceptionally it may follow an individual word. Footnote numbers start with 1. Examples in footnotes have the numbers (i), (ii), etc.

\subsection{Tables and figures}

Tables and figures are treated as floats in typesetting. This means that their placement on the page will not necessarily be where you put them in your manuscript, as this may lead to large parts of the page ending up white (e.g. when a table or figure does not fit on the current page anymore and wraps onto the following page). For this reason, you must always refer to tables and figures in the running text, as in the following example: ``In certain languages, the superlative transparently contains the comparative morphologically, as illustrated in table \ref{tbl:table1} \citep[46]{Bobaljik2012}.'' Do not refer to tables and figures using the words ``following'', ``below'' or ``above'', as the final placement of your table or figure may be different from where you placed them in your manuscript.

\begin{table}[h]
\begin{tabular}{lllll}
 & \textsc{Pos} & \textsc{Cmpr} & \textsc{Sprl}\\
\toprule
Persian & kam & kam-tar & kam-tar-in & `little’\\
Cimbrian & šüa & šüan-ar & šüan-ar-ste & `pretty’ \\
Czech & mlad-ý & mlad-ší & nej-mlad-ší & `young’\\
Hungarian & nagy & nagy-obb & leg-nagy-obb & `big’\\
Latvian & zil-ais & zil-âk-ais & vis-zil-âk-ais & `blue’\\
Ubykh &  nüs\textsuperscript{w}\textipa{@} & ç’a-nüs\textsuperscript{w}\textipa{@} & a-ç’a-nüs\textsuperscript{w}\textipa{@} & `pretty’ \\
\bottomrule
\end{tabular}
\caption{Morphological containment}
\label{tbl:table1}
\end{table}

Tables and figures are numbered consecutively. Each table and each figure has a caption. The caption is placed below figures and tables, with the figure or table number in bold.  If the caption is not a complete sentence, it is not followed by a period. Examples are shown in the captions of table \ref{tbl:table1} and figure \ref{fig:glossalogo}.

\begin{figure}[h]
\includegraphics[width=10cm]{glossa}
\caption{The Glossa logo (design by Linnea Vanden Wyngaerd)}
\label{fig:glossalogo}
\end{figure}

Figures should be included in the main text for the purpose of peer review. Once the paper is accepted, all figures must be uploaded separately as supplementary files, if possible in colour and at a resolution of at least 300dpi. No file should be larger than 20MB. Standard formats accepted are: \textsc{jpg, tiff, gif, png, eps}. For line drawings, please provide the original vector file (e.g. .ai, or .eps).

Tables must be created using a word processor's table function, not tabbed text. Tables should be included in the manuscript.

Tables should not include:

\begin{itemize}
\item Rotated text
\item Colour to denote meaning (it will not display the same on all devices)
\item Images
\item Diagonal lines
\item Multiple parts (e.g. ``table 1a'' and ``table 1b''). These should either be merged into one table, or separated into ``table 1'' and ``table 2''.
\end{itemize}
If there are more columns than can fit on a single page, the table will be rotated by 90 degrees to fit on the page. Do not use tables that cannot fit onto a single page.

Tree diagrams should be treated as examples, not as figures. If your figure or tree diagram includes text, then for the best match with the typeset text use the font \href{https://software.sil.org/charis/download/}{Charis SIL}, or \href{https://www.fontsquirrel.com/fonts/fira-sans}{Fira Sans}. These fonts also support the International Phonetical Alphabet (IPA) symbols.


\subsection{In-text citations}

The short reference form used in the text consists of the author’s surname and the publication year, followed by page numbers where necessary. Brackets surround the year, except if the citation is already inside brackets, in which case there are no brackets around the year. If there are more than two authors, the surname of the first author plus \textit{et al.} can be used. If all the authors are listed, they are all separated by an ampersand.

\begin{itemize}
\item \citet[514]{murray:1983} point out that \ldots
\item The notation we use to represent this is borrowed from theories according to which $\phi$-features occur in a so-called feature geometry \citep[248-250]{mccarthy:1999}.
\item Baker et al. (1989) = \citet*{baker:1989}
\end{itemize}
When multiple citations are listed, they are separated by semicolons and listed in chronological order. Multiple references to the same author do not repeat redundant information.

\begin{itemize}
\item Multiple authors have belaboured this point \citep{chomsky:1981,chomsky:1986a,chomsky:1986,iverson:1989,casali:1998a,blevins:2004,franks:2005}.
\end{itemize}
Surnames with internal complexity have upper or lower case according to how the author spells his/her own name, e.g.:

\begin{itemize}
\item It has been claimed by \citet{swart:1998} and \citet{belder:2011} that meaning is compositional.
\end{itemize}
Chinese and Korean names may be treated in a special way: as the surnames are often not very distinctive, the full name may be given in the in-text citation, e.g.

\begin{itemize}
\item  \ldots the neutral negation \textit{bù} is compatible with stative and activity verbs (cf. Teng Shou-hsin 1973; Hsieh Miao-Ling 2001; Lin Jo-wang 2003) %to achieve this in latex, list the author in your bib-file with brackets around firstname+lastname, e.g. {Teng Shou-hsin}
\end{itemize}

\subsection{References}\label{sec:refs}

The following rules apply:

\begin{itemize}
\item The names of authors and editors should be given in their full form as in the publication, without truncation of given names.
\item All author and editor names are given in the order ``Lastname, Firstname''.
\item When there is more than one author (or editor), each pair of names is separated by an ampersand.
\item Page numbers of journals are obligatory (issue numbers preferred).
\item Journal titles are not abbreviated.
\item Main title and subtitle are separated by a colon, not by a period.
\item No author names are omitted, i.e. et al. is not used in the references.
\end{itemize}

There are four standard reference types: journal article, book, article in edited book, thesis. Works that do not fit easily into these types should be assimilated to them to the extent that this is possible. See the bibliography at the end of this article for examples.

Surnames with internal complexity are never treated in a special way. Thus, Dutch or German surnames that begin with \textit{van} or \textit{von} (e.g. van Riemsdijk) or French and Dutch surnames that begin with with \textit{de} (e.g. de Saussure) are alphabetized under the first part, even though they begin with a lower-case letter. Thus, the following names are sorted alphabetically as indicated:

\begin{itemize}
\item Da Milano, Federica
\item de Groot, Casper
\item De Schutter, Georges
\item de Saussure, Ferdinand
\item van der Auwera, Johan
\item Van Langendonck, Willy
\item van Riemsdijk, Henk
\item von Humboldt, Wilhelm
\end{itemize}

Capitalise all lexical words (title case) in journal titles and titles of book series. Capitalise only the first word (plus proper names and the first word after a colon) for book and dissertation titles, and article and chapter titles. The logic is to use title case for the titles that are recurring, lower case for those that are not.

Important note for \LaTeX\ users: when typesetting the bibliography with the Glossa style files, titles of articles, chapters, books, and dissertation titles will automatically have their capitals made lowercase, e.g. if your bib-file has `Passive Arguments Raised' as an article title, this will be typeset as `Passive arguments raised' in the references list, since this is what the stylesheet wants. However, this procedure has the unwanted side-effect that words that should keep a capital also lose it, most notably the names of languages. For example, a title like `VSO versus VOS: Aspects of Niuean word order' will appear as `Vso versus vos: Aspects of niuean word order'. In order to avoid this from happening, you must in your bib-file protect words with necessary capitals by surrounding them with braces, like this:  `\{VSO\} versus \{VOS\}: Aspects of \{Niuean\} word order'.

Names of book series are optional; they directly follow the book title, without intervening punctuation. They appear between brackets, have title case, and Roman font. They may be accompanied by an (optional) issue number.

Titles of works written in a language that readers cannot be expected to know should be accompanied by a translation, given in brackets \citep{Li1999}.

Glossa style in Citation Style Language (CSL) for use with Zotero is available \href{https://www.zotero.org/styles?q=Glossa}{here}. Many thanks to Mark Dingemanse for creating this style, and to Lisa Levinson for updating it.


\subsection{Typographical matters}

\subsubsection{Capitalisation}

Sentences, proper names and titles/headings/captions start with a capital letter, but there is no special capitalisation (`title case') within English titles/headings, neither in the article title nor in section headings or figure captions. Capitalisation is also used after the colon in titles, i.e. for the beginning of subtitles. Capitalisation in the references section follows its own rules (see section \ref{sec:refs}).

%Capitalisation is used only for parts of the article (sections, figures, tables, appendixes) when they are numbered, e.g.
%\begin{itemize}
%\item as shown in Table 5
%\item more details are given in Section 3
%\item this is illustrated in Figure 17
%\end{itemize}

Please refrain from the use of FULL CAPS (except for abbreviations).

\subsubsection{Italics}
Italics are used in the following cases:
\sloppy
\begin{itemize}
\item for technical terms and all object-language forms (letters, words, phrases, sentences) that are cited within the text, unless they are phonetic transcriptions or phonological representations in IPA.
\item for emphasis within the text of a particular word that is not a technical term.
\item for emphasis within a quotation, with the indication [emphasis mine/ours] at the end of the quotation.
\item for the name of the language in examples.
\end{itemize}

In numbered examples, do not use italics to highlight particular parts of the example; use bold instead.

\fussy

\subsubsection{Small caps}
%\paragraph{Small caps} Small caps are used for grammatical categories in the interlinear glosses in examples (e.g. \textsc{fut, neg, sg, obl}, etc.). They are also used for indicating stressed syllables or words in example sentences.

Small caps are used for grammatical categories in the interlinear glosses in examples (e.g. \textsc{fut, neg, sg, obl}, etc.). They are also used for indicating stressed syllables or words in example sentences.

\subsubsection{Boldface and other highlighting}
Boldface can be used to draw the reader’s attention to particular aspects of a linguistic example, whether given within the text or as a numbered example. Full caps, underlining, or italics are not normally used for highlighting.

\subsubsection{Quotation marks}
Double quotation marks are used

\begin{itemize}
\item when a passage from another work is cited in the text.
\item when a technical term or other expression is mentioned that the author does not want to adopt.
\end{itemize}
Ellipsis in a quotation is indicated by [\ldots].

Single quotation marks are used exclusively for linguistic meanings, e.g.
\begin{itemize}
\item Latin \textit{habere} ‘have’ is not cognate with Old English \textit{hafian} ‘have’.
\end{itemize}
Quotes within quotes are not treated in a special way.
Note that quotations from other languages should be translated (inline if they are short, in a footnote if they are longer).

\subsubsection{Abbreviations}
When a complex term that is not widely known is referred to frequently, it may be abbreviated (e.g. DOC for ``double-object construction''). The abbreviation should be given in the text when it is first used. Abbreviations of uncommon expressions are not used in headings or captions, and they should be avoided at the beginning of a chapter or major section.

The abbreviations used in glossed examples should all be listed in a separate section following the conclusions. For a list of standard abbreviations, refer to the \href{https://www.eva.mpg.de/lingua/resources/glossing-rules.php}{Leipzig glossing rules}.



\section{Submission preparation checklist}

As part of the submission process, authors are required to check off their submission's compliance with all of the following items, and submissions may be returned to authors that do not adhere to these guidelines.

\begin{enumerate}[label=\arabic*.]
\item The submission has not been previously published, nor is it being considered for publication by another journal (or an explanation has been provided in Comments to the Editor).
\item Any third-party-owned materials used have been identified with appropriate credit lines, and permission obtained from the copyright holder for all formats of the journal.
\item All authors have given permission to be listed on the submitted paper and satisfy the \href{https://www.glossa-journal.org/site/authorship/}{authorship guidelines}.
\item The submission is provided as \textbf{a single PDF file}, containing all tables and figures
\item All \textsc{doi}s for the references have been provided, when available.
\item Tables and figures are all provided in the submitted PDF and correctly cited in the text.
\item Figures/images have a resolution of at least 300dpi and are included in the main manuscript file being submitted.
\item The author(s) agree to edit their text to adhere to the stylistic and bibliographic requirements outlined in the \href{https://www.glossa-journal.org/about/submissions#authorGuidelines}{Author Guidelines}, should the paper be editorially accepted.
\item All references to the author(s) have been removed from the paper. Aside from omitting the author’s name in the title block, this entails only referring to your own work in the third person (do not use ‘Author 1’ or a similar replacement for your own name). Also check the acknowledgments and the funding information sections for identifying information.
\item Author names have been removed from the document properties of the manuscript file (check the File menu of your PDF software for document properties).
\item The maximum word count has been adhered to. See Author Guidelines for more information.
\item For review purposes, make sure that your contribution has page numbers.
\item Please email \href{paula.clementevega@openlibhums.org}{paula.clementevega@openlibhums.org}, should a waiver be required, or an author come from an \href{https://www.openlibhums.org/plugins/supporters/}{OLH supporting institution}.
\end{enumerate}

\section{Copyright notice}

Authors who publish with this journal agree to the following terms:

\begin{enumerate}[label=\arabic*.]
\item Authors retain copyright and grant the journal right of first publication with the work simultaneously licensed under a \href{https://creativecommons.org/licenses/by/4.0/}{Creative Commons Attribution License} (CC)BY 4.0) that allows others to share the work with an acknowledgement of the work's authorship and initial publication in this journal. (See \href{http://opcit.eprints.org/oacitation-biblio.html}{The Effect of Open Access}.)
\item Authors are able to enter into separate, additional contractual arrangements for the non-exclusive distribution of the journal's published version of the work (e.g., post it to an institutional repository or publish it in a book), with an acknowledgement of its initial publication in this journal.
%\item Authors are permitted and encouraged to post their work online (e.g., in institutional repositories or on their website) prior to and during the submission process, as it can lead to productive exchanges, as well as earlier and greater citation of published work .
\end{enumerate}

\section{Privacy Policy}

The privacy policy can be viewed \href{https://www.openlibhums.org/site/privacy/}{here}.

\section{Publication fees}

Authors publishing in Glossa face no financial obligation for the publication of their article. Authors from institutions that already have an  \href{https://www.openlibhums.org/plugins/supporters/}{OLH membership} will have the full Article Processing Charge (APC) covered by the consortium of libraries participating in the \href{https://www.openlibhums.org}{Open Library of Humanities} (OLH), ensuring long-term sustainability. We recommend that authors from non-member institutions ask their libraries to support OLH with annual contribution that will cover any current / future publication in the journal. Should a submitting/corresponding author be from an institution that already has an \href{https://www.openlibhums.org/plugins/supporters/}{OLH membership}, please indicate it accordingly when submitting your paper.

Authors from OLH non-member institutions that have access to funds earmarked for APCs (via a research grant or through their institution) will be asked to use those funds to cover the £450 APC of their publication in Glossa. \textbf{Authors without access to such funds will be asked to request a waiver through the submission system.} This full APC waiver will then be logged against the submission.

The APC covers all publication costs (editorial processes; web hosting; indexing; marketing; archiving; DOI registration etc.) and ensures that all of the content is fully open access. This approach maximises the potential readership of publications and allows the journal to be run in a sustainable way.

If you do not know about your institution’s policy on open access funding, please contact your departmental/faculty administrators and institution library, as funds may be available to you.

Shortly after publication, authors that have not already requested a waiver from OLH will receive an APC request email along with information on how payment can be arranged. If the APC situation has changed since submission to publication, an APC waiver can also be requested at this point.

If you have any questions, please email \href{mailto:paula.clementevega@openlibhums.org}{paula.clementevega@openlibhums.org}.

\section{Conclusion}

The conclusion is the last numbered section, and any ensuing sections are unnumbered.

\section*{Abbreviations (if applicable)}\label{abbrev}

\textsc{acc} = accusative, \textsc{dat} = dative, \textsc{dem} = demonstrative, \textsc{nom} = nominative, \textsc{pl} = plural, \textsc{sg} = singular

For the standard abbreviations to be used here, refer to the \href{https://www.eva.mpg.de/lingua/resources/glossing-rules.php}{Leipzig glossing rules}.

\section*{Data availability/Supplementary files (if applicable)}

The journal encourages authors to make all data associated with their submission openly available, according to the FAIR principles (Findable, Accessible, Interoperable, Reusable). More information can be found \href{https://www.glossa-journal.org/site/editorial-policies/#data-policy}{here}.

If data/supplementary files are to be associated with the accepted paper, one of the options below should be followed:
\begin{enumerate}
\item upload the files to your chosen open repository and make note of the DOI that they will provide (most suitable for datasets or information that act as foundations to the research being published. This option makes the files more findable and more citable). We recommend an open repository such as osf.io, which allows you to create a "project" under which you can upload relevant files (datasets, analysis scripts, experimental materials, etc.). The project will be associated with a unique DOI. You can then include in your manuscript a citation of the OSF entry and/or a link to the project page on OSF, to direct interested readers to the supplementary materials. During review, please be sure that the link to the repository is anonymized to maintain a fully double masked review process. Instructions for doing this on the OSF may be found \href{https://help.osf.io/hc/en-us/articles/360019930333-Create-a-View-only-Link-for-a-Project}{here}. If you'd like to learn more about best practices for ensuring reproducibility, see \href{https://psyarxiv.com/hf297/}{Laurinavichyute and Vasishth (2021)}. Please contact us if you would like more information or advice about hosting your data on an open repository.
\item upload the files to the journal system during the submission process, as `data files'. The journal will then host them as part of the publication and provide them with a DOI (most suitable for non-data files or very short pieces of information, although option 1 is also suitable for these if the author prefers).
\end{enumerate}

\noindent In both cases, a `Data availability' or `Supplementary files' section must be added prior to the reference list that provides a title and very short summary of the files for each file. If option 1 was selected, you should also provide the DOI in this section. For example:

\noindent Supplementary file 1: Appendix. Scientific data related to the experiments. DOI: \doi{10.5334/gjgl.310.s1}

Ideally, supplementary files are also cited in the main text.

Please note that neither of the above two options will result in the files being typeset, so please ensure that they are in publishable format when you upload the accepted paper.


\section*{Ethics and consent (if applicable)}

Research involving human subjects, human material, or human data, must have been performed in accordance with the Declaration of Helsinki. Studies must have been approved by an appropriate ethics committee and the authors should include a statement in the article text detailing this approval, including the name of the ethics committee and reference number of the approval, or mention any exemptions to ethical approval that apply to their research. The identity of research subjects should be anonymised whenever possible. For research involving human subjects, informed consent to participate in the study must be obtained from participants (or their legal guardian).


\section*{Funding information (if applicable)}

Should the research have received a funding grant then the grant provider and grant number should be detailed.

\section*{Acknowledgements (optional)}

The authors wish to thank Martin Haspelmath for providing the generic style sheet for linguistics, and Kai von Fintel for giving permission to use and modify the \textit{Semantics \& Pragmatics} Latex template, bibliography style, and document class.

\section*{Competing interests (required)}

If any of the authors have any competing interests then these must be declared. Guidelines for competing interests can be found \href{https://www.glossa-journal.org/site/competing-interests/}{here}. If there are no competing interests to declare then the following statement should be present: `The author(s) has/have no competing interests to declare'.

\section*{Authors' contributions (optional)}\label{contrib}

A sentence or a short paragraph detailing the roles that each author held to contribute to the authorship of the submission.  Individuals listed must fit within the definition of an author, as per our \href{https://www.glossa-journal.org/site/author-guidelines/}{Author Guidelines}.

\nocite{*} %this is to get all the entries of the sample bibliography; delete this line for an actual Glossa submission

%\printbibliography %for use with biblatex; comment out if you use natbib
\bibliography{sample} %for use with natbib; comment out if you use biblatex, and change 'sample' by the name of your bib-file


\end{document}
%%% Local Variables:
%%% mode: latex
%%% TeX-master: t
%%% TeX-engine: luatex
%%% End:
